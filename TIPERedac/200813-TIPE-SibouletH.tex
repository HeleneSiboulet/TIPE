\documentclass[11pt,a4paper]{article}
\usepackage[francais]{babel}
\usepackage[utf8]{inputenc}
\usepackage[T1]{fontenc}

\usepackage{amsmath}
\usepackage{amsfonts}
\usepackage{amssymb}
\usepackage{makeidx}
\usepackage{graphicx}
\usepackage[left=1cm,right=1cm,top=1cm,bottom=2cm]{geometry}
\usepackage{tabularx}
\usepackage{epstopdf}
\usepackage{booktabs}
\usepackage{threeparttable}
\newcommand{\Angstrom}{\textup{\AA}}
\usepackage{longtable}
\usepackage{multirow}
%\usepackage[labelfont=bf]{caption}
\usepackage{multicol}
\usepackage{pbox}
%\usepackage{hyperref}
%\usepackage{cleveref}
%\usepackage[pdftex]{graphicx}
\usepackage{epstopdf}
\usepackage{float}
\usepackage{rotating}
\usepackage{array}
 
%\usepackage{lmodern}
%\usepackage{mathpazo}
%\usepackage{microtype}
%\usepackage{geometry}
\usepackage{caption}
%\usepackage{xkeyval}
%\usepackage{subfigure}
\usepackage[dvipsnames]{xcolor}
\usepackage{sectsty}
%\usepackage{selinput}
\usepackage{babel}
%\usepackage{ulem}
%\usepackage{mathtools}
\usepackage{framed}
%\graphicspath{{/Users/bertrandsiboulet/Documents/minutes/figures/}}
\DeclareGraphicsExtensions{.eps,.pdf}
 
\newcolumntype{?}{!{\vrule width 2pt}}
%\usepackage[toc,page]{appendix}
\chapterfont{\color{NavyBlue}}
%\titleformat{\chapter}[display]
%{\normalsize \huge \color{NavyBlue}}%
%{\flushright \normalsize \color{Black}}%
%{\MakeUpperCase{\chaptertitlename}\hspace{1ex}%
%{\fontfamiliy{mdugm}\fontsize{60}{60}\selectfont\thechapter}}%
%{10pt}%
%{\bfseries\huge}%
\sectionfont{\color{BrickRed}}
\subsectionfont{\color{Maroon}}
\subsubsectionfont{\color{Red}}
\paragraphfont{\color{OrangeRed}}
%\renewcommand{\familydefault}{\sfdefault}
%\usepackage{geometry}
%\usepackage{pgfplots}
\makeatletter
\renewcommand\paragraph{\@startsection{paragraph}{4}{\z@}%
{-2.5ex\@plus -1ex \@minus -.25ex}%
{1.25ex \@plus .25ex}%
{\normalfont\color{OrangeRed}\normalsize\bfseries}}
\makeatother
 
\setcounter{secnumdepth}{4}
\setcounter{tocdepth}{4}
\bibliographystyle{elsarticle-num}
 
\usepackage[automark]{scrpage2}
\setheadsepline{.4pt}
\usepackage{listings}
\makeatletter
\renewcommand\tableofcontents{%
    \@starttoc{toc}%
}
\makeatother
\renewcommand{\arraystretch}{1.2}
%commands used in my current template
\newcommand*\mycommand[1]{\texttt{\emph{#1}}}
\newcommand*{\doi}[1]{\href{http://dx.doi.org/#1}{doi: #1}}
\newcommand{\comment}[1]{\textit{\textcolor{Red}{#1}}}
\newcommand{\addref}[1]{\textit{\textcolor{Red}{Add reference!}}}
\newcommand{\sub}[1]{\textsuperscript{#1}}
 
 
\begin{document}
\begin{flushleft}
\Large \textcolor{BrickRed}{\textsc{\textbf{Travaux d'Initiative Personnelle Encadrés}}} \\
\Huge \textcolor{BrickRed}{\textsc{\textbf{Informatique et prévisions météorologiques }}} \\[0.5cm]
\Large \textcolor{BrickRed}{\textsc{\textbf{Concours CPGE 2021 \\[0.5 cm]}}}
\hrule \textcolor{White}{-} \\[0.1cm]
%\Large \textcolor{BrickRed}{\textsc{\textbf{Concours CPGE 2021 \\}}}
\Large \textsc{Hélène Siboulet MP} \\
\Large \textsc{Juin 2021} \\[0.5cm]
\hrule
\end{flushleft}
\section{Présentation} 
%%%%%%%%%%
La prévision consiste à prévoir, c'est-à-dire à \og voir avant\fg{}. Il s'agit donc, à partir de ce que l'on sait à un moment donné, de savoir ce qui se passera à un moment ultérieur, avec la meilleure probabilité.En effet, à part les phénomènes déterministes, comme par exemple le mouvement des solides à court terme, on ne peut avoir que des indications probabilistes. C'est un problème universel, qui s'applique à l'économie, la politique, la vie privée, et aussi bien sûr aux prévisions météorologiques.   \\
La prévision repose sur deux éléments : des données disponibles à un moment, et des théories ou des modèles qui permettent d'utiliser ces données. \\
Les processus météorologiques sont tout à fait chaotiques. Cela signifie que des incertitudes minimes à un instant donné peuvent avoir des conséquences très fortes sur le déroulement. En fait, ces incertitudes n'empêchent pas de prévoir à court terme, par exemple pour le lendemain, mais elles rendent difficiles les prévisions au delà de deux semaines. \\
Si les processus météorologiques sont chaotiques à moyen terme, ils sont au contraire extrêmement stables dès lors que on les analyse en moyenne. Par exemple, Milankovitch a prévu que les variations de trois paramètres~-excentricité, obliquité et précession~-auraient des effets périodiques sur les glaciations. Milankovitch a fait apparaître des périodes de 20~000, 40~000 et 100~000 ans, ce qui a été confirmé expérimentalement par les mesures de $\delta O_{18}$ dans les carottes glaciaires de Vostock, en 1976. Ces résultats montrent que, en faisant des moyennes sur une centaine d'année, on a des tendances très claires d'évolution des températures, alors qu'il est difficile de prévoir la température dans deux semaines.   
 
Dans ce documents, nous allons faire un bref état de l'art des méthodes de prévision météorologiques utilisées, et proposer une méthode basée sur l'intelligence artificielle. Nous verrons que ces deux approches sont tout à fait différentes. Nous collectons une base de données, nous élaborons une méthode avec de nombreuses variantes. Chaque variante est évaluée numériquement, c'est à dire par comparaison des prévisions avec un ensemble de données qui ont été placées en dehors des données d'entrées. 

 \vspace {0.6cm}
Les prévisions météorologiques ont beaucoup d'enjeux, elles permettent notamment :
\begin {enumerate}
\item aux pilotes d'avion et aux marins de savoir si leur trajet est réalisable, de connaitre la quantité de carburant à prendre et de choisir l'itinéraire le plus adapté
\item aux agriculteurs de savoir quand arroser, quelle quantité d'eau est nécessaire et quand labourer, planter ou récolter 
\item d'organisation d'activités ou d'évènements en plein air
\item de gérer des risques liés au climat par exemple lorsqu'il y a du verglas, de la neige, une canicule ou une tempête
\end{enumerate}

 
\section{État de l'art}
%Méthode la plus courante de prévision météo selon Jean Pailleux de Météo-France
%Modélisation de l'atmosphère par :
%\begin{enumerate}
%\item Équation du mouvement(Newton)
%\item $\frac{d\bold{V}}{dt} = \bold(g) - \frac{\bold{grad}p}{d} $
%\item Équation de continuité
%\item Thermodynamique
%\item Équation des gaz parfaits
%\item Équations de bilans de constituants: vapeur d’eau, eau liquide, ozone, etc...
%\end{enumerate}

Il existe plusieurs sortes de prévisions météorologiques : la prévision du temps qui fait des prédictions sur des périodes de quelques jours, la prévision saisonnière qui font des prédiction sur l'échelle d'un moi ou de quelques mois et la prévision climatiques prédit l'évolution du climat sur plusieurs années.

Selon http://meteocentre.com/intermet/prevision/prevision\_emp.htm
les principales méthodes de prévision météorologiques sont :
\begin{enumerate}
\item La méthode de la persistance qui prédit que demain il fera le même temps qu'aujourd'hui. Elle fonctionne bien dans certaines régions où les conditions météorologiques varient lentement comme	l'Egypte ou les déserts des États-Unis. Elle fait de bonnes prévisions sur des périodes de un ou deux jours ainsi et est efficace pour prédire des tendance sur des durées d'un moi (prévisions saisonnière). 
\item La méthode de la tendance qui étudie les déplacement de systèmes météorologiques tels que les dépressions, les anticyclones, les fronts et les zones de précipitation et suppose leurs emplacement futurs.
\item La méthode de l'analogie qui consiste à chercher des cas dans le passé qui ressemblent aux conditions météorologiques actuelles et à supposer que ces conditions évolueront de la même manière.
\item La méthode numérique est la plus utilisée actuellement. On divise l'atmosphère dans une grille en 3 dimensions. On mesure les différentes paramètres. Les lois physiques prédisent l'évolution du système.
\end{enumerate}

On trouve sur internet d'autre types d'outils de prévision tels que des réseaux de neurones, des algorithmes génétiques, de la logique floue, des réseaux bayésiens et d'autres modèles probabilistes voir Das2017.

La méthode numérique 

\section{Recherche de données}

Nous avons trouvé une base de donnée sur  : \\
https://donneespubliques.meteofrance.fr/?fond=produit\&id\_produit=90\&id\_rubrique=32
Elle contient les relevés de 62 stations en France Métropolitaine et en France d’Outre mer
de 1996 à 2020  (avec certaines mesures manquantes : pas plus de 80 en une année soit l'équivalent de dix jours sauf pour 2020)
avec un relevé toutes les trois heures soit huit relevés par jour.
Les donnees sont : la température, l'humidité, la direction et la force du vent, la pression atmosphérique, la hauteur de précipitations, le temps sensible, la description des nuages, la visibilité.
Chaque fichier couvre un mois et toutes les stations

Collecte des données avec Webbot
Décompression des fichiers au format csv
Tri des données par station

\section{Analyse sur une station unique}

Extraction des données de température et d'humidité de Clermont-Ferrand au au format json
Les annees 2008, 2019 et 2020 sont retirées de la base d'entrainement et serviront de base de test\\
! donnes\_collecte/meteo\_france/analyse\_station\_unique/humidite.png\\
! donnes\_collecte/meteo\_france/analyse\_station\_unique/temperature.png\\

Comment estimer la valeur d'un ensemble de prévisions~? On calcule l'écart quadratique moyen des écarts entre nos prévisions et les températures mesurées. cela permet d'estimer la valeur des prévisions, mais à condition d'avoir une valeur de comparaison. Nous utilisons pour cette valeur les diffréences entre les quartiles 1 et 3. Calcul des Quartiles de l'humidité et de la température. En dessous du premier quartile, on trouve le quart des valeurs, et au dessus du troisièpe quartile, on trouve un quart des valeurs. 
Température : Q1 = 6.39  Q3 = 17.4  soit un intervalle de 11${}^{\circ}C$  
Humidité : 	  Q1 = 59.0  Q3 = 85.0  soit un intervalle de 26\%

* Calcul de la moyenne de la température et de l'humidité par jour et heure
Approximation de la température à un phoénomène cyclique de période un an

ecart quadratique moyen de temperature : 4.28${}^{\circ}C$
ecart quadratique moyen d'humidite : 14.8\%

* Approximation de la courbe des températures par un fonction
À priori est un phoénomène cyclique sur l'année et sur le jour
Transformée de Fourier

Hypothèse vérifiée :
Approximation par une fonction de la forme $f(t) = A cos (w1 t + \phi^1) + B cos (w2 t + \phi^2) + C$
avec $w1 = 2 \pi  et w2 = 2 \pi /365$
Et par une fonction de la forme $g(t) = A cos (w1 t + \phi^1) + B cos (w2 t + \phi^2) + C + D cos (w3 t + \phi^3)$
avec $w3 = 4 \pi$

On utilise la méthode du gradient pour trouver les coefficient :
\begin{itemize}
\item on crée une fonction écart $(u(t))$ qui calcule l'écart entre les prévisions de $u(t)$ et les valeurs de la base d'entrainement
\item on cherche à trouver le minimum de écart en fonction des paramètres donc
\item on initialise aléatoirement les paramètres
\item on dérive écart par rapport à chaque paramètre
\item on modifie chaque paramètre X en faisant X = X - dX
\item on répète les deux points précédents un grand nombre de fois
\end{itemize}


\section{Exemples Latex}
\subsection{subsection}
\subsubsection{q;,c}
%Donner un référence : \cite{Adler:1964aa}
\subsubsection{Donner des figures}
\begin{figure}
\centering
\includegraphics[width=0.48 \textwidth]{/Users/helene/Documents/TIPE/donnees_collecte/meteo_france/analyse_station_unique/temperature.png}\quad
\includegraphics[width=0.48 \textwidth]{/Users/helene/Documents/TIPE/donnees_collecte/meteo_france/analyse_station_unique/temperature.png}
\caption{\label{fig:190101Lolita} A droite : pointe AFM / surface d'eau, à gauche :  Coalescence de gouttes d'eau }
\end{figure}
%%%%%%
\begin{figure}
  \includegraphics[width=0.48 \textwidth]{/Users/helene/Documents/TIPE/TIPERedac/imagesTIPE/temperature.png}\quad
\end{figure}
\begin{figure}
  \includegraphics[width=0.48 \textwidth]{/Users/helene/Documents/TIPE/TIPERedac/imagesTIPE/temperature.png}\quad
\end{figure}
%%%%%
%%%\begin{figure}
%\centering
%\includegraphics[width=0.48 \textwidth]{/Users/siboulet/Desktop/imagesTIPE/temperature.png}\quad
%\includegraphics[width=0.48 \textwidth]{/Users/siboulet/Desktop/imagesTIPE/humidite.png}
%\caption{\label{fig:190101Lolita} A droite : pointe AFM / surface d'eau, à gauche :  Coalescence de gouttes d'eau }
%\end{figure}

\subsubsection{Citer des figures}
Je cite la figure \ref{fig:190101Lolita}
%%%%%
\subsection{Placer un tableau}
\begin{table}[ht]
\begin{tabular}{cccc}\hline
\hline
Ow& 15.99945& -0.8476&SPCE\\
Hw&  1.0079 &  0.4238&SPCE\\
Na& 22.9897 &  1.0000&\\         
Cl& 35.45   & -1.0000&\\
Os& 15.9994 & -1.000 &\\   
Ti& 47.8670 &  2.000 &\\  
\hline 
\end{tabular}                     
\caption{\label{MonTableau}, Charges for Ti02/W/Na/Cl}
\end{table}
\subsection{Citer un tableau}
Je cite mon tableau \ref{MonTableau}.


\section{Références}
%\bibliography{/Users/helene/Documents/TIPE/TIPERedac/bibH}
%\begin{enumerate}
%\item Baselov, I., Vasileva, D., 2016. Numerical modeling of drop coalescence in the presence of soluble surfactants. J. Computational and Applied Mathematics 293, 7-19.
%\item Duchemin, L., Eggers, J., Josserand, C., 2003. Inviscid coalescence of drop. J. Fluid Mech. 487, 167-178.
%\end{enumerate}

\end{document}